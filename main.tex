
\documentclass[titlepage,10pt]{article}
%
%
%

\usepackage[letterpaper,top=2cm,bottom=2cm,left=3cm,right=3cm,marginparwidth=1.75cm]{geometry}


\usepackage{orcidlink}                % orcid package
\usepackage[affil-it,blocks]{authblk} % author list package


\begin{document}
\title{Tile of your physics proposal \\(PAC53)} % EDIT to be your proposal name


%
% This LaTeX author list works with the PAC53 proposal template and can also be used to import your author information into the PAC submittion form.
%
% Please use full names
% Please make sure to note the contact-spokesperson with $^{*\dagger}$ and any co-spokespersons with $^{*}$
% Please add the orcid number when possible
% Please add a commented email so people can be added to the JLab proposal author database 
%

%
% Example Author List
%

\author{Douglas~Higinbotham$^{*}$~\orcidlink{0000-0003-2758-6526}} % email: doug@jlab.org
\author{Tyler~Hague$^{\dagger}$~\orcidlink{0000-0003-1288-4045}} % email: tyler@jlab.org
\author{Mark~M.~Dalton$^{*}$~\orcidlink{0000-0001-9204-7559}} % email: dalton@jlab.org
\affil{Thomas Jefferson National Accelerator Facilty, Newport News, VA, USA}

\author{Full~Name~~\orcidlink{0000-0000-0000-0000}} % email: xxx@yyy.zzz
\author{Full~Name~$^{*}$~\orcidlink{0000-0000-0000-0001}} % email: xxx@yyy.zzz
\author{Full~Name~\orcidlink{0000-0000-0000-0002}} % email: xxx@yyy.zzz
\author{Full~Name~\orcidlink{0000-0000-0000-0003}} % email: xxx@yyy.zzz
\affil{Name, Address, Country}

%
\date{\small{*: Co-spokesperson, $\dagger$: Contact-spokesperson (add email address here)}}
%
 % EDIT the authors.tex file with your proposals author information

\maketitle

\section*{Executive Summary}

\newpage

\section{Introduction}

Provide a general introduction to experiment and the physics that this experiment is trying to address.   Through-out this document being concise and clear is extremely helpful to the reader and help people accurate judge the merits of the proposal.    

\section{Physics Motivation}

This should you should provide the theoritical and/or phenomological reason for performing this measurement.

\section{Experimental Details}

In this section you should provide about the experiment.   For standard equipment, you can just provide references.  For example, here is a reference for the parameters of 
the CEBAF 12GeV accelerator ~\cite{Adderley:2024czm} and it also an example of how to use bibtex.  

\subsection{Required Equipment}

If using standard equipment this can be short, BUT please be sure to note details such as length of target cells and other custom requirement.   Details about any new detectors should be provided here.

\subsection{Statistics and Systematics}

Provide details about the both the statistic and systematic uncertainties expected in the measurement.    The level of detail required will depend on if this is a measurement with statistical uncertainties that are far larger then the systematics or if this is a systematics limited measurement.

\subsection{Kinematics}

For the large acceptance devises such as CLAS or GlueX this section should provide the field details but for the spectrometer halls, please provided both momentums and angles that will be needed.   These details are important to make sure there is no interfearance or other limitations that would make getting to the experiments desired kinematics.    

\section{Beamtime Request} 

Beamtime request should be made assuming the accelerator avaliable at all times: an efficency factor is applied later in the procees.    


\section{Summary}

Provide a short, high-level summary of the proposal and how this experiments help improve our understanding of the topic and conclude with the number of PAC days being requested.

%
%
%


%
% References.    
%
% It is recommended to use bibtex for references.    The inspirehep.net site has most of the papers and the by clicking "cite" provides bibtex formatted citation information. 
%
%
\newpage
\bibliographystyle{unsrt}
\bibliography{references}
\end{document}

